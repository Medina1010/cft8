\begin{problem}{Librería en C++ para Operaciones con Números Complejos}{1}
  {\bf Introducción} \\

  Los números complejos son esenciales en diversas áreas de la física computacional, como mecánica cuántica y transformadas de Fourier. El objetivo de esta tarea fue implementar una clase en C++ que permita realizar operaciones básicas con números complejos y convertir dicho código en una librería utilizable desde Python.

  {\bf Metodología} \\

  Primero se implementó la clase {\it ComplexNumber} en C++, con métodos para suma, resta, multiplicación, división, conjugado y módulo. Luego el código se compiló como librería dinámica.

  Finalmente, se creó un wrapper usando SWIG para importar la librería desde Python y ejecutar pruebas básicas.

  {\bf Resultados} \\

  El código y documentación se encuentra en el repositorio de github de este informe: \url{https://github.com/Medina1010/cft8.git}.

  {\bf Discusión} \\

  El ejercicio permitió comprobar la correcta organización del código en C++, la creación de librerías y la interoperabilidad con Python. La modularidad del código y la separación en archivos facilitaron la compilación y el uso de la librería.

  {\bf Conclusión} \\

  Se logró implementar una librería funcional para números complejos en C++ y utilizarla desde Python. El proyecto refuerza conceptos fundamentales de programación estructurada, librerías dinámicas e integración entre lenguajes, elementos clave en la física computacional.

\end{problem}

